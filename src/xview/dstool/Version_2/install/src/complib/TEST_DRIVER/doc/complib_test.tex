\documentstyle{article}


% File of macros and standard definitions
% -------------------------------------------------------------------------------

% Definition for bitmap images to be included 
\def\sunbitmap#1#2#3#4{%
 \hfill \hbox to #1{%
  \vbox to #2{
   \vss
   \special{pre: sunbitmap #3 #40}
   \special{ps: sunbitmap #3 #40}
  }%
 }%
 \hfill
}

% Basic directory commands
%\newcommand{\PATH}{c:\mark\texfiles}
%\newcommand{\GRAPHICS}{c:\mark\texfiles}

% Basic LATEX macros
\newcommand{\be}{\begin{equation}}
\newcommand{\ee}{\end{equation}}
\newcommand{\ben}{\begin{displaymath}}
\newcommand{\een}{\end{displaymath}}
\newcommand{\qed}{
   \begin{flushright}
     {\mbox{$\Box$}}\end{flushright}\vspace{.25in}}

% set special fonts
\font\setfont=eufm10

% Standard math macros
\newtheorem{theorem}{Theorem}[section]
\newtheorem{corollary}{Corollary}[section]
\newtheorem{prop}{Proposition}[section]
\newcommand{\proof}{{\bf Proof:} }
\newcommand{\real}{{\mbox{I} \hspace{-.03in} {\bf R} }}
\newcommand{\realf}{\mbox{{\bf R}$^3$}}
\newcommand{\plane}{\mbox{{\bf R}$^2$}}
\newcommand{\realn}{\mbox{{\bf R}$^n$}}
\newcommand{\dstool}{\mbox{\bf dstool}}
\newcommand{\gameq}{\mbox{$\Gamma$-equivariant }}
\newcommand{\vf}{\mbox{$x^{\prime}=F^Q(x,\lambda)$}}
\newcommand{\pvf}{\mbox{$u^{\prime}=P^Q(u)$}}
\newcommand{\group}{\mbox{\setfont G}}
\newcommand{\property}{\mbox{\setfont P}}
\newcommand{\checkoff}{\mbox{\bf \large $\surd$}}
\newcommand{\blsp}[1]{\makebox[.28in]{$ #1 $}}
\newcommand{\algsp}{\rule[.05in]{0.0in}{.15in}}
%\newcommand{\property}{\mbox{P}}

% set margins to 1 inch + specified quantity
% ------------------------------------------
\oddsidemargin 0.25in
\topmargin 0.25in
\headheight 0.25in

\textwidth 6in
\headsep 0in
\textheight 8.25in

% end of standard definitions
% -------------------------------------------------------------------------------



\begin{document}

\begin{center}
{\LARGE \bf Complib\_Test Propagation Program}  \\

\vspace{.15in}
{\large \bf Subsection Test and Verification Driver}
\end{center}

\vspace{.2in}
\begin{tabbing}
00000000000000000000\=00000000012345678\=00000123\=0123456789\=0123456789\=0123456789\=0123456789\=0123456789 \kill
\> {\bf Authors: }\> dstool Software Group \\
\> \>        Center for Applied Mathematics  \\
\> \>        Cornell University \\
\> \>        305 Sage Hall \\
\> \>        Ithaca, New York \hspace{.1in}  14853 \\
\> \>          \>   \\
\>{\bf Version: }\> 1.0 \hspace{.07in}  (Last Change: August 1992)   \\
\>             \>   \\
\>{\bf Bug Reports: } \> dstool\_bugs@macomb.tn.cornell.edu
\end{tabbing}
\vspace{.2in}

\noindent {\large \bf Purpose:}  The {\em complib\_test} utility provides a standalone driver for
the propagation subsection of the dstool software package.
In construction, the main procedure performs the role of
a level 2 routine (see dstool PDS documentation), which initializes the data structure 
used to pass data to the iteration routines (iterated maps) or integration
algorithms (vector fields).  This program can emulate any propagation calculation which can
be performed by dstool and, in addition, allows the programmer or algorithm designer the
opportunity to study the implementation of a new routine apart from the interactive windowing
system so that properties such as convergence and timing behavior can be accurately established.
\vspace{.15in}

\section{Overview:}

\noindent The complib\_test main program serves as a Level-2 interface routine to the propagation 
subsection of dstool;  see the dstool Reference Manual \cite{ref1} for more details on the organization
of the propagation routines collected together in the {\em complib.a} subroutine library for dstool.
The program complib\_test and dstool share the same computation engine, and
complib\_test executes precisely the same computation routines in exactly the same way as dstool.
It is composed of a main driver, {\em driver.c}, and a collection
of enumerated program files which specify individual tests which are used to exercise the integration
and iteration code. 
The distribution version of the complib\_test program has been pre-configured to include
eight problem testcases selected from a variety of sources in the literature;   however,
the real utility of complib\_test should become evident when the dstool user installs problem
files which pertain to the dynamical systems of their research interests.
\medskip

\noindent To create the complib\_test program, merely change the current directory to that which contains
the complib\_test source and invoke the UNIX make utility:
\begin{tabbing}
00000000000000000000\=00000000012345678\=00000123\=0123456789\=0123456789\=0123456789\=0123456789\=0123456789 \kill
\> cd \$\{DSTOOL\}/src/computation/complib/TEST\_DRIVER \\
\> make complib\_test
\end{tabbing}
at the UNIX prompt;  the file called Makefile contains a complete set of instructions for creating an executable
version from the source code.  Once this step is complete, complib\_test may be called to execute any one of
several pre-installed example testcases, selected using the following command-line arguments:
\bigskip

\hspace{.25in}\parbox[b]{.5in}{-p}
\parbox[t]{4.6in}{
Program identification number.  The next Section of this document provides 
a description of each problem that comes installed in the distribution version.
A particular problem is specified by typing its identification number following 
the -p option flag.  If no identification number is provided, complib\_test will 
execute the default problem,  the integration of a linear vector field. 
}
\bigskip

\hspace{.25in}\parbox[b]{.5in}{-m}
\parbox[t]{4.6in}{ 
Integration method. If the problem defined is the propagation of a flow for a
vector field, the integration method is selected using this command-line 
argument.  Any integration method installed in dstool may be specified;  for the
current distribution release, fixed-step $4^{th}$-order Runge-Kutta (option 0),
Euler's method (option 1), variable-step $4^{th}$-order Runge-Kutta (option 2) 
and Bulirsch-Stoer (option 3) are allowed choices.
}
\bigskip

\hspace{.25in}\parbox[b]{.5in}{-a}
\parbox[t]{4.6in}{  
Execute all allowed integration methods on the specified test problem.  This option 
has no meaning if the specified problem is an iterated mapping.
}
\bigskip
\bigskip

\noindent For example, the following command would execute the $4^{th}$ installed testcase
problem using the variable-step Runge-Kutta integration algorithm:
\begin{tabbing}
00000000000000000000\=00000000012345678\=00000123\=0123456789\=0123456789\=0123456789\=0123456789\=0123456789 \kill
\> complib\_test -p 4 -m 2
\end{tabbing}
while the command:
\begin{tabbing}
00000000000000000000\=00000000012345678\=00000123\=0123456789\=0123456789\=0123456789\=0123456789\=0123456789 \kill
\> complib\_test 
\end{tabbing}
would produce a sequence of integrate points for the default vector field using the fixed-step Runge-Kutta
technique, a trajectory for the differential equation $\dot{y} \! = \! -y$, as described below.



\section{Installed Problems}

\subsection{Default Problem}

\noindent The default problem is executed when the {\em complib\_test} program
is executed without specifying an installed problem using the `-p' option.  The
defining equations are those prescribed by Hull, et al. \cite{hull1} as testcase (A1)
in  the DETEST collection.  We state our problem setup as though it were to be executed
with the default integrator -- the fixed-step $4^{th}$-order Runge-Kutta algorithm:

\medskip
\begin{center}
\fbox{
\parbox{4.6in}{ 
\rule{4.6in}{0in}
\begin{tabbing}
000\=0000000\=00000000000000000000\=0000000000000000000000000000000000000000000000000000000\=00000\= \kill
\> {\bf Given:} \\
\> \> $
       \begin{array}{l}
           \dot{y}  =  -y \rule{.5in}{0in} t \in [0,\frac{3}{2}] \\
           y(0)  = 1 
       \end{array} $ \\
\> \\
\> {\bf Find:} \\
\> \> $y(t_f) \; \; \mbox{with} \; \; t_f = (150 \; \mbox{steps}) \cdot (0.01 \; \mbox{time units per step} )$ \\ 
\>  \\
\> {\bf Exact Solution:} \\
\> \>          $y(t) = \mbox{\normalsize $e$}^{-t}$
\end{tabbing}
}}
\end{center}
\medskip

\noindent Initiating execution of the {\em complib\_test} problem with no options specified
will propagate an orbit in the flow of this vector field using the fixed-step Runge-Kutta
($4^{th}$ order) algorithm with the following result:

\begin{verbatim}

   Integration Method:  Runge-Kutta (4) 
    
   Iter    Stepsize     time        y-Computed        y-Exact        Difference
   ----    --------    ------       ----------        ---------     ------------ 
   0                     0           1                1             0  
   1         0.01        0.01        0.9900498        0.9900498     8.318901e-13  
   2         0.01        0.02        0.9801987        0.9801987     1.647349e-12  
   3         0.01        0.03        0.9704455        0.9704455     2.446376e-12  
   4         0.01        0.04        0.9607894        0.9607894     3.229417e-12  
   5         0.01        0.05        0.9512294        0.9512294     3.996581e-12
\end{verbatim}
\vspace{-.1in}
\hspace{2.8in}$\vdots$
\vspace{-.1in}
\begin{verbatim}
   145       0.01        1.45        0.2345703        0.2345703     2.858133e-11  
   146       0.01        1.46        0.2322363        0.2322363     2.84921e-11  
   147       0.01        1.47        0.2299255        0.2299255     2.840181e-11  
   148       0.01        1.48        0.2276377        0.2276377     2.831049e-11  
   149       0.01        1.49        0.2253727        0.2253727     2.821818e-11  
   150       0.01        1.5         0.2231302        0.2231302     2.812492e-11
\end{verbatim}



\subsection{Problem 1}

\noindent  This ordinary differential equation defines a logistic curve and appears as testcase (A4)
in the DETEST collection.   The complib subsection is capable of numerically integrating a flow
to a fixed value of the independent variable, for each of the algorithms installed in the 
distribution version of dstool.  We illustrate this capability by propagating the trajectory with
initial condition $y(0) = 1$ to the final condition $y(t_f)$ where $t_f=1.5$ time units.

\medskip
\begin{center}
\fbox{
\parbox{4.6in}{ 
\rule{4.6in}{0in}
\begin{tabbing}
000\=0000000\=00000000000000000000\=0000000000000000000000000000000000000000000000000000000\=00000\= \kill
\> {\bf Given:} \\
\> \> $
       \begin{array}{l}
           \dot{y}  =   \frac{y}{4} \left ( 1 - \frac{y}{20} \right )    \rule{.5in}{0in} t \in [0,\frac{3}{2}] \\
           y(0)  = 1 
       \end{array} $ \\
\> \\
\> {\bf Find:} \\
\> \> $y(t_f) \; \; \mbox{with} \; \; t_f = 1.5 \; $ time units \\ 
\>  \\
\> {\bf Exact Solution:} \\
\> \>	   $y(t) = { \mbox{\rule[-.02in]{0in}{.16in} $20$ } \over
		{\mbox{\rule{0in}{.12in} $ 1 + 19 \mbox{\normalsize $e$}^{-t/4} $ }} } $ 

\end{tabbing}
}}
\end{center}
\medskip



\subsection{Problem 2}

\noindent  This 4-dimensional system of ordinary differential equations is used by Press, et al. \cite{press1,vetter1}
to illustrate the fundamental differences between the $4^{th}$-order fixed stepsize Runge-Kutta, variable-step
Runge-Kutta and Bulirsch-Stoer algorithms as described in the text {\em Numerical Recipes in C}.  

\medskip
\begin{center}
\fbox{
\parbox{4.6in}{ 
\rule{4.6in}{0in}
\begin{tabbing}
000\=0000000\=00000000000000000000\=0000000000000000000000000000000000000000000000000000000\=00000\= \kill
\> {\bf Given:} \\
\> \> $
       \begin{array}{l}
           \dot{y}_1 \rule[-.1in]{0in}{.2in}  =  -y_2 \rule{.5in}{0in} t \in [1,20]  \\
           \dot{y}_2 \rule[-.1in]{0in}{.2in}  =  y_1 - { 1 \over \mbox{\normalsize $t$} } y_2  \\
           \dot{y}_3 \rule[-.1in]{0in}{.2in}  =  y_2 - { 2 \over \mbox{\normalsize $t$}  } y_3  \\
           \dot{y}_4 \rule[-.1in]{0in}{.2in}  =  y_3 - { 3 \over \mbox{\normalsize $t$}  } y_4  
       \end{array} $ \\
\> \\
\> {\bf Find:} \\
\> \> $y(t_f) \; \; \mbox{where} \; \; t_f = 20.0 \;$ time units \\ 
\end{tabbing}
}}
\end{center}
\medskip




\subsection{Problem 3}

\noindent  This nonlinear chemical reaction problems was considered by Frost \cite{frost1}, and included in
the DETEST collection as testcase (B3). 


\medskip
\begin{center}
\fbox{
\parbox{4.6in}{ 
\rule{4.6in}{0in}
\begin{tabbing}
000\=0000000\=00000000000000000000\=0000000000000000000000000000000000000000000000000000000\=00000\= \kill
\> {\bf Given:} \\
\> \> $
       \begin{array}{l}
           \dot{y}_1 \rule[-.1in]{0in}{.2in}  =  -y_2 \rule{.5in}{0in} t \in [1,20]  \\
           \dot{y}_2 \rule[-.1in]{0in}{.2in}  =  y_1 - y_2^2  \\
           \dot{y}_3 \rule[-.1in]{0in}{.2in}  =  y_2^2  \\
	   y_1(0) = 3  \; \; \; y_2(0) = y_3(0) = 0
       \end{array} $ \\
\> \\
\> {\bf Find:} \\
\> \> $y(t_n) \; \;$ where $n$ equal $150$ and $h_i \! = \! t_{i+1} \! - \! t{i}$ \\
\> \>  is the integration stepsize (fixed or variable). \\
\end{tabbing}
}}
\end{center}
\medskip


\subsection{Problem 4}

\noindent  This example is derived from Van der Pol's equation, 
\[
   \ddot{y} - \left ( 1-y^2 \right ) \dot{y} + y = 0
\]
and appears as testcase (E2) in the DETEST collection \cite{hull1}.  We note that for a
different choice of parameters this vector field is stiff, and appears in a collection
used to evaluation numerical methods for stiff systems of ODE's \cite{enright1}.  


\medskip
\begin{center}
\fbox{
\parbox{4.6in}{ 
\rule{4.6in}{0in}
\begin{tabbing}
000\=0000000\=00000000000000000000\=0000000000000000000000000000000000000000000000000000000\=00000\= \kill
\> {\bf Given:} \\
\> \> $
       \begin{array}{l}
           \dot{y}_1 \rule[-.1in]{0in}{.2in}  =  y_2 \rule{.5in}{0in} t \in [1,20]  \\
           \dot{y}_2 \rule[-.1in]{0in}{.2in}  =  (1 - y_1^2)y_2 - y_1  \\
	   y_1(0) = 2 \rule{.2in}{0in} y_2(0) = 0 \\
       \end{array} $ \\
\> \\
\> {\bf Find:} \\
\> \> $y(t_n) \; \;$ where $n$ equal $150$ and $h_i \! = \! t_{i+1} \! - \! t{i}$ \\
\> \>  is the integration stepsize (fixed or variable). \\
\end{tabbing}
}}
\end{center}
\medskip


\subsection{Problem 5}

\noindent  The phase portrait for the integrable Hamiltonian system with Hamiltonian, 
\[
     {\cal H}\left ( y_1,y_2 \right ) = { y_2^2 \over 2 } - { y_2^2 \over 2 } +  { y_1^4 \over 4 }
\]
contains a homoclinic orbit based at the equilibrium point, $(\sqrt{2},0)$.  The analytic 
solution for this connecting orbit may be obtained by elementary methods in terms of
the hyperbolic trigonometric  functions, {\em sech()} and {\em tanh()}.   This invariant 
set for the vector field separates two very different regions of behavior in the phase
portrait: One open set foliated with periodic solutions and a second containing 
unbounded trajectories.  The separatrix for this problem represents the (non-transversal)
intersection of the stable and unstable manifolds for the fixed point at $\; y^* \! = \! (0,0)$.
\medskip

\noindent We use this example to exercise two particular features of the propagation computation engine:
backward integration (in the independent variable, $t$) and terminating the orbit propagation based
on the crossing of a surface of section defined by a user-supplied function.

\medskip
\begin{center}
\fbox{
\parbox{4.6in}{ 
\rule{4.6in}{0in}
\begin{tabbing}
000\=0000000\=00000000000000000000\=0000000000000000000000000000000000000000000000000000000\=00000\= \kill
\> {\bf Given:} \\
\> \> $
       \begin{array}{l}
           \dot{y}_1 \rule[-.1in]{0in}{.2in}  =  y_2 \rule{.5in}{0in} t \! \leq \! 0  \\
           \dot{y}_2 \rule[-.1in]{0in}{.2in}  =  (1 + y_1^2)y_1  \\
	   y_1(0) = \sqrt{2} \rule{.2in}{0in} y_2(0) = 0 \\
       \end{array} $ \\
\> \\
\> {\bf Find:} \\
\> \> $y(t_f) \; \;$ such that $y(t_f)$ lies in the ball $ \; ||y^* \! - \! y(t_f)||$\raisebox{-.08in}{\scriptsize 2}$ \! < \! 0.01$  \\ 
\> \> \hspace{.38in}  of the fixed point $ \; y^* = (0,0)$.
\>  \\
\> {\bf Exact Solution:} \\
\> \> $
       \begin{array}{l}
           y_1(t) \rule[-.1in]{0in}{.2in}  =  \sqrt{2} \;  \mbox{sech} (t)  \\
           y_2(t) \rule[-.1in]{0in}{.2in}  =  - \sqrt{2} \; \mbox{sech}(t) \; \mbox{tanh}(t)  \\
       \end{array} $ 
\end{tabbing}
}}
\end{center}
\medskip




\subsection{Problem 6}

\noindent This problem is used by Dahlquist \cite{dahlquist1} to demonstrate the instability
of the fixed-step Runga-Kutta method when applied to a stiff ordinary differential equation.

\medskip
\begin{center}
\fbox{
\parbox{4.6in}{ 
\rule{4.6in}{0in}
\begin{tabbing}
000\=0000000\=00000000000000000000\=0000000000000000000000000000000000000000000000000000000\=00000\= \kill
\> {\bf Given:} \\
\> \> $
       \begin{array}{l}
           \dot{y}  =  100 \cdot ( \sin t - y ) \rule{.5in}{0in} t \in [0,\frac{1}{3}] \\
           y(0)  = 0 
       \end{array} $ \\
\> \\
\> {\bf Find:} \\
\> \> $y(t_f) \; \; \mbox{with} \; \; t_f = 3.0 \;$ time units \\ 
\>  \\
\> {\bf Exact Solution:} \\
\> \>          $y(t) = { \mbox{\rule[-.02in]{0in}{.16in} $ \sin t - 0.01 \cdot \cos t +
			    0.01 \cdot \mbox{\normalsize $e$}^{-100 t}$ } \over
		       {\mbox{\rule{0in}{.12in} $ 1.0001 $ }} } $
\end{tabbing}
}}
\end{center}
\medskip

\noindent The relevant control structure parameters for this problem are the integration time step, the
termination mode and the final time:

\begin{verbatim}
   prop_cntl->prop_mode  = PROP_TF;    /* propagate until final_time is reached */
   prop_cntl->final_time = 3.0;        /* desired final integration time */
   prop_cntl->time_step  = 0.030;      /* UNSTABLE stepsize for RK4 !! */
   prop_cntl->time_step  = 0.025;      /* stable stepsize for RK4 */
\end{verbatim}



\subsection{Problem 7}

\noindent  The complib propagation subsection may be used to iterate a mapping as well as integrate
a vector field;  this example serves to illustrate its capability in the case of the {\em standard
map}, a diffeomorphism defined on the 2-torus, $T^2$.  The initial conditions selected for the
distribution example is for a period-5 periodic orbit.

\medskip
\begin{center}
\fbox{
\parbox{4.6in}{ 
\rule{4.6in}{0in}
\begin{tabbing}
000\=0000000\=00000000000000000000\=0000000000000000000000000000000000000000000000000000000\=00000\= \kill
\> {\bf Given:} \\
\> \> $
       \begin{array}{l}
          f(x,y)  = \left ( x - \omega - y \; , \; {1 \over \mbox{\normalsize $b$}} \left (
		       y + {\mbox{\normalsize $k$} \over \mbox{\normalsize $2 \pi$}}
		       \sin(2 \pi x) \right )  \right ) \\
          \omega = 0.0 \hspace{.15in} k = 0.97 \hspace{0.15in} b = 1.0 
       \end{array} $ \\
\> \\
\> {\bf Find:} \\
\> \> $(x_n,y_n) = f^{15}(x_0,y_0)  \; \; \mbox{where} \; \; (x_0,y_0) = (0.5,0.646696840644)$ \\ 
\>  \\
\> {\bf Exact Solution:} \\
\> \> $ (x_{15},y_{15}) = (x_0,y_0) $
\end{tabbing}
}}
\end{center}
\medskip


\clearpage

\begin{thebibliography}{99}

\bibitem{AGK}
D.~Armbruster, J.~Guckenheimer, and S.~Kim
\newblock Chaotic dynamics in systems with square symmetry.
\newblock Physics Letters A, 140, 1989.

\bibitem{AVK}
A.~Andronov, A. Vitt, and S. Khaiken.
\newblock {\it Theory of Oscillations}.
\newblock New York, Pergamon, 1966.

\bibitem{arnold}
V.~I.~Arnold.
\newblock {\em Mathematical Methods of Classical Mechanics}, 2nd Ed.
\newblock Springer-Verlag, New York, 1989.

\bibitem{BGKM}
C.~Baesens, J.~Guckenheimer, S.~Kim, and R.~Mackay.
\newblock Three coupled oscillators: mode-locking, global
bifurcations, and toroidal chaos.
\newblock Physica D 49, 1991.

\bibitem{BGKM:IMA}
C.~Baesens, J.~Guckenheimer, S.~Kim, and R.~Mackay.
\newblock Simple resonance regions of torus diffeomorphisms.
\newblock IMA preprint series \#656, 1990.

\bibitem{BC} 
M. Benedicks and L. Carleson.
\newblock The dynamics of the H\'{e}non map.
\newblock Annals of Mathematics, 133, 1991.

\bibitem{chirikov}
B. V. Chirikov.
\newblock A universal instability of many-dimensional oscillator
systems.
\newblock Physics Reports, 52, 1979.

\bibitem{dangle1}
G.~Danglemayr and J.~Guckenheimer
\newblock On a Four Parameter Family of Planar Vector Fields
\newblock Archive for Rational Mechanics and Analysis, Vol. 97, No. 4, 1987, pps. 321-352.

\bibitem{golubitsky1}
P.~Chossat and M.~Golubitsky. 
\newblock Symmetry increasing bifurcation of chaotic attractors. 
\newblock Physica D 32 (1988).

\bibitem{Deng:constructing}
B.~Deng.
\newblock Constructing homoclinic orbits and chaos.
\newblock Preprint, April 1991.

\bibitem{Devaney}
R.~Devaney.
\newblock {\em An Introduction to Chaotic Dynamical Systems}.
\newblock  Addison-Wesley, Redwood City, CA, 1989.

\bibitem{epsteinmarder}
I. R. Epstein and E.~Marder.
\newblock Multiple modes of a conditional neural oscillator
\newblock Biol. Cybern. 63, 1990.

\bibitem{golubitsky2}
M. Field and M. Golubitsky.
\newblock Symmetric chaos.  
\newblock Computers in Physics.   Sep./Oct. 1990.

\bibitem{golubitsky3} 
M.~Golubitsky and D. G.~Schaeffer.
\newblock {\it Singularities and groups in bifurcation theory}.
\newblock Applied Mathematical Sciences Series, Vol. 51, Springer-Verlag,  1985.

\bibitem{gucken1} J.~Guckenheimer
\newblock Multiple bifurcation problems for chemical reactors.
\newblock Physica D 20, 1986.

\bibitem{gucken2}
J.~Guckenheimer
\newblock Dynamics of the Van der Pol Equation
\newblock IEEE Transactions on Circuits and Systems, Vol. CAS-27, No. 11, 1980, pps. 983-989.

\bibitem{GGHW}
J. Guckenheimer, S. Gueron and R. Harris-Warrick.
\newblock The dynamics of a conditionally bursting neuron.
\newblock Preprint, 1992.

\bibitem{GH}
J.~Guckenheimer and P.~J. Holmes.
\newblock {\em Nonlinear Oscillations, Dynamical Systems, and Bifurcations of
  Vector Fields}.
\newblock Springer-Verlag, New York, 1983.

\bibitem{GuckenMahalov}
J.~Guckenheimer and A.~Mahalov.
\newblock Phys. Rev. Lett. 68, 1992.

\bibitem{GuckenheimerWorfolk:instant}
J.~Guckenheimer and P.~Worfolk.
\newblock Instant chaos.
\newblock Nonlinearity, to appear, 1992.

\bibitem{Henon:mapping}
M.~H\'{e}non.
\newblock A two-dimensional mapping with a strange attractor.
\newblock {\em Comm. in Mathematical Physics}, 50:69--77, 1976.

\bibitem{hh1}
A.L.~Hodgkin and A.F. Huxley
\newblock {A Quantitative Description of Membrane Current and its Applications
  to Conduction and Excitation in Nerve}. 
\newblock Journal of Physiology, Vol. 117, 1952, pps. 500-544.

\bibitem{kubicek1} 
M.~Kubi\u{c}ek
\newblock Evaluation of complex bifurcation points.
\newblock SIAM Journal of Applied Mathematics, Vol. 38, No. 1, 1980.

\bibitem{lorenz1}
E.N.~Lorenz
\newblock Deterministic Non-Periodic Flows
\newblock Journal of Atmospheric Science, Vol. 20, 1963, pps. 130-141.

\bibitem{MoraViana:abundance}
L.~Mora and M.~Viana.
\newblock Abundance of strange attractors.
\newblock preprint, 1991.

\bibitem{rinzel1} 
R.J.~Rinzel and Y.S. Lee
\newblock Dissection of a model for neuronal parabolic bursting.
Journal of Mathematical Biology, Vol. 25, 1987.

\bibitem{rinzel2}
J.~Rinzel and R.N. Miller
\newblock {Numerical calculation of stable and unstable periodic solutions 
	  to the Hodgkin-Huxley equations}.
\newblock Mathematical Biosciences, Vol. 49, 1980, pps. 27-59.

\bibitem{salamsastry} 
F.~Salam and S.~Sastry.
\newblock Dynamics of the forced Josephson junction circuit: the
regions of chaos.
\newblock IEEE Transactions on Circuits and Systems, CAS-32, 8, 1985.

\bibitem{sparrow1}
C.~Sparrow
\newblock {\em The Lorenz Equations: Bifurcations, Chaos, and Strange Attractors}.
\newblock Springer-Verlag Series in Applied Mathematical Sciences, Vol. 41, 1982.

\bibitem{takens1}
F.~Takens
\newblock Forced Oscillations and Bifurcations
\newblock {Communications of the Mathematical Institute, Rijkuniversiteit Utrecht,
           Vol. 3, 1974, pps. 1-59.}

\bibitem{uppal1} 
A.~Uppal, W. H.~Ray, and A.~B.~Poore.
\newblock On the dynamic behavior of continuous stirred tank reactors.
\newblock Chemical Engineering Science, Vol. 29, 1974.

\bibitem{vdp1}
B.~van der Pol
\newblock {Forced Oscillations in a Circuit with Nonlinear Resistance
(Receptance with Reactive Triode)}.
\newblock London, Edinburgh and Dublin Phil. Mag., Vol. 3, 1927, pps. 65-80.

\end{thebibliography}


\end{document}

