\section{Hodgkin-Huxley Equations}

\noindent The Hodgkin/Huxley equations are an empirically derived system of four
nonlinear differential equations which describe the electrical response of  the giant nerve
axon from the squid {\em Loligo} to an externally-applied current \cite{hh1}.  The typical 
response of the cell to a step in the stimulus current, $I$, is characterized by
an abrupt spike in the electrical potential 
difference, $v$, between the intracellular fluid and the extracellular medium called an
{\em action potential}.   In the Hodgkin/Huxley model this depolarization is induced primarily by an inward flux of sodium ($Na^+$) followed by
an outward flow of potassium ($K^+$) ions. Other ions contribute to
a ``leak'' current across the
axon membrane. The sodium and potassium currents are controlled 
by three {\em gating} variables denoted
$m$, $n$ and $h$, together with parameters, 
$g_{na}$, $g_k$ and $g_l$ that measure the maximum conductances of
the channels.  The resulting vector field is given by:

\begin{equation}
\label{hheq1}
\begin{array}{lcl}
\dot v &=&-G(v,m,n,h) - I\\
\dot m &=&\Phi (T) \left[(1-m)  \alpha _m(v) - m \beta _m(v)\right]\\
\dot n &=&\Phi (T) \left[(1-n)  \alpha _n(v) - n \beta _n(v)\right]\\
\dot h &=&\Phi (T) \left[(1-h)  \alpha _h(v) - h \beta _h(v)\right]
\end{array}
\end{equation}
where $\dot x$ stands for $dx/dt$ and $\Phi$ is given by
$\Phi (T)=3^{(T-6.3) / 10}$. The other functions involved are:
$$G(v,m,n,h)=\bar{g}_{\rm na} m^3h(v-\bar{v}_{\rm na}) + \bar{g}_{\rm k}
n^4(v-\bar{v}_{\rm k}) + \bar{g}_{\rm l}(v-\bar{v}_{\rm l})
$$
and the equations modeling the variation of membrane permeability:
$$\begin{array}{lclclcl}
\alpha _m (v)&=&\Psi ({{v+25} \over 10})
 & \rule{.2in}{0in} 
&\beta _m (v)&=&4 e^{v /18}
\\
\alpha _n (v)&=&\mbox{$\frac{1}{10}\Psi ({{v +10}\over 10})$ \rule{0in}{.2in} }
 & \rule{.2in}{0in} 
&\beta _n (v)&=&\mbox{$\frac{1}{8} e^{v/{80}}$}
\\
\alpha _h (v)&=&\mbox{$\frac{7}{100}e^{v / 20}$ \rule{0in}{0.2in} }
 & \rule{.2in}{0in} 
&\beta _h (v)&=&\left( 1+e^{(v+30)/ 10}\right)^{-1}
\end{array}$$

$${\rm with}\qquad \Psi (x)=\left\{ \begin{array}{ll}
x /(e^x-1) & {\rm if } \  x \ne 0\\
1& {\rm if} \  x= 0
\end{array} \right. \rule{0.4in}{0in} $$
Except where otherwise noted,  we use the temperature
$T=6.3^\circ$C and
parameter values for $\bar g_{\rm ion}$  and
$\bar{v}_{\rm ion}$ used by Hodgkin and Huxley \cite{hh1}:

$$\begin{array}{lll}
	\bar{g}_{\rm na}= 120 {\rm m.mho}/{\rm cm}^2	
&\bar{g}_{\rm k}= 36{\rm m.mho}/ {\rm cm}^2 
&\bar{g}_{\rm l}= 0.3 {\rm m.mho}/{\rm cm}^2\\
	\bar{v}_{\rm na}= -115 {\rm mV}	
&\bar{v}_{\rm k}= 12 {\rm mV}	
&\bar{v}_{\rm l}=10.599 {\rm  mV}
 \end{array}$$

\noindent Evidence that the space-clamped Hodgkin-Huxley equations exhibit stable periodic solutions
that  arise through Hopf bifurcation for certain regions in the $v_k \! - \! I$ parameter space
has been well established (for example, see \cite{rinzel2} and references therein).
