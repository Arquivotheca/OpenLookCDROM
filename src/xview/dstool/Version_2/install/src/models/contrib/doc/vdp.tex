\section{Forced van der Pol Oscillator}

\noindent An important example of oscillator with nonlinear damping
is provided by the Van der Pol equations, introduced by B. Van der Pol \cite{vdp1}
as a model of an a triode value electrical circuit with variable
current-dependent resistance.  The defining equations are given by:

\begin{eqnarray*}
   \dot{x} & = & y  \\
   \dot{y} & = & \alpha (1-x^2) y - x + \beta \cos(\omega t)     
\end{eqnarray*}

\noindent An important feature present 
in the unforced $(\beta=0)$ dynamical system is
a {\em relaxation} oscillation near a unique limit cycle. 
The file {\bf vdp1} shows this limit cycle;  the initial conditions
provided may be used to initiate a trajectory which illustrates
the properties of the flow near the limit cycle.  The orbits will
move very slowly near the vertical  segments of the limit cycle and
then rapidly jump along the horizontal portions.
\medskip

\noindent The behavior of the system is much more complicated when the
time-varying forcing term is turned on;  indeed, a full description of
the dynamics for all values of the parameters remains an open problem \cite{gucken2}.






