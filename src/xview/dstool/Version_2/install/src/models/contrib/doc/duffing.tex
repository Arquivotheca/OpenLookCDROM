\section{Duffing's Equation}

Duffing's equation describes the motion of a buckled beam when
only one mode of vibration is considered \cite{GH}.
It is given by the following  vector field:
\begin{eqnarray*}
	u' & = & v \\
	v' & = & \beta u - u^3 - \delta v + \gamma cos(\theta) \\
	\theta ' & = & \omega
\end{eqnarray*}

The file {\tt duffing1.dat} contains initial conditions 
for a regime where a chaotic attractor exists.  Try varying
the parameter $\delta$ to observe how the chaotic attractor evolves.
\begin{enumerate}
\item $\delta = 0.15$ A strange attractor and large stable period 1 orbit coexist.
\item $\delta = 0.20$ All orbits are attracted to a strange attractor.
\item $\delta = 0.22$ A stable period 3 orbit exists.
\end{enumerate}
Look at unforced dynamics by setting $\gamma = 0$.

The file {\tt duffing2.dat} sets up a Poincar\'{e} section which can be
used to study the structure of the chaotic attractor.
We suggest that you change to a quality control integrator like
Runge-Kutta 4QC or Bulirsch-Stoer.  Examine the same attracting
objects by varying $\delta$ as before.
