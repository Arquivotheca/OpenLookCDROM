\section{Standard Mapping}

The so-called ``standard mapping'' was introduced in an unpublished
work by J. B. Taylor (1968) and was extensively studied in a lengthy survey
paper by Chirikov \cite{chirikov}.  It is the canonical description of
a particle moving in a sinusoidal potential which is turned one for an
infinitesimally brief time every unit of time.  That is, the particle
receives a periodic ``kick'' or impulse, the strength of which depends
on the particle's position.

Often the standard map is studied in the context of Hamiltonian
dynamics (that is, the mapping is {\em symplectic} or area
preserving).  In {\bf DsTool}, the equations defining the standard
mapping include a term for adding dissipative effects to the model.
The equations are
\begin{eqnarray*}
  x' &=& x + \omega + y' \\
  y' &=& b y - \frac{k}{2 \pi} \sin(2 \pi x).
\end{eqnarray*}
The variable $x$ is periodic with period 1, so the phase space for the
standard mapping is the cylinder $S^1 \times R$.  The parameter $b$
controls the dissipation, with $b=1$ corresponding to a symplectic
mapping. 

The dynamics of this mapping shows a complex structure.  For $k \ll 1$
the dynamics occur on smooth KAM (Kolmogorov-Arnold-Moser) curves or
extremely thin stochastic bands. As $k$ increases, resonant ``islands''
become apparent.  These islands are surrounded by a ``sea'' of
stochasticity.  As $k$ approaches the value 1, the last KAM surface is
destroyed and an initial condition within the stochastic sea can come
arbitrarily close to any $y$ value, that is, there are no longer any
impenetrable barriers to global transport of phase space particles.